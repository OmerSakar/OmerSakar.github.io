\cvsection{Publications}

\begin{cventries}
	

\cventry
{Formal Methods in System Design} % Degree
{Preserving provability over GPU program optimizations with annotation-aware transformations. } % Institution
{-} % Location
{2025} % Date(s)
{
    \begin{cvitems} % Description(s) bullet points
        \item Authors: Ömer Şakar, Mohsen Safari, Marieke Huisman, and Anton Wijs.
        \item Abstract: GPU programs are widely used in industry. To obtain the best performance, a typical development process involves the manual or semi-automatic application of optimizations prior to compiling the code. Such optimizations can introduce errors. To avoid the introduction of errors, we can augment GPU programs with (pre- and postcondition-style) annotations to capture functional properties. However, keeping these annotations correct when optimizing GPU programs is labor-intensive and error-prone.
        This paper presents an approach to automatically apply optimizations to GPU programs while preserving provability by defining annotation-aware transformations. It applies frequently-used GPU optimizations, but besides transforming code, it also transforms the annotations. The approach has been implemented in the Alpinist tool and we evaluate Alpinist in combination with the VerCors program verifier, to automatically apply optimizations to a collection of verified programs and reverify them.
    \end{cvitems}
}


\cventry
{International Conference on Software Engineering and Formal Methods 2024} % Degree
{Deductive Verification of SYCL in VerCors.} % Institution
{Aveiro, Portugal} % Location
{November 4-8, 2024} % Date(s)
{
    \begin{cvitems} % Description(s) bullet points
        \item Authors: Ellen Wittingen, Marieke Huisman, and Ömer Şakar. 
        \item Abstract: SYCL is a C++ programming model for the development of heterogeneous programs. It uses the concept of kernels, where multiple instances of a computation are executed concurrently on a computing unit. This concurrency entails that the set of possible program behaviours can be of considerable size, which makes these programs error-prone. Formal verification could be used to ensure the correctness of all these possible program behaviours. However, there exist no formal verification tools for SYCL.
        In this paper, SYCL support is added to VerCors, a formal verification tool for concurrent software, by encoding SYCL constructs into VerCors’ internal language COL. To the extent of our knowledge, this is the first deductive verification tool for SYCL. We show how SYCL’s basic- and ND-range kernels are encoded, along with the encoding and challenges related to scheduling kernels and the execution order of those kernels. In addition, we discuss how SYCL’s buffers and data accessors are encoded, focusing on the challenges related to it, in particular enabling memory transfer between host and device. The usability of the added SYCL support and how it was evaluated are discussed as well.
        \item DOI: \url{https://doi.org/10.1007/978-3-031-77382-2_11}
    \end{cvitems}
}

    
\cventry
{International Conference on Computer-Aided Verification 2024} % Degree
{The VerCors verifier: a progress report} % Institution
{Montreal, Canada} % Location
{July 22–27, 2024} % Date(s)
{
    \begin{cvitems} % Description(s) bullet points
        \item Authors: Lukas Armborst, Pieter Bos, Lars B van den Haak, Marieke Huisman, Robert Rubbens, Ömer Şakar, and Philip Tasche.
        \item Abstract: This paper gives an overview of the most recent developments on the VerCors verifier. VerCors is a deductive verifier for concurrent software, written in multiple programming languages, where the specifications are written in terms of pre-/postcondition contracts using permission-based separation logic. In essence, VerCors is a program transformation tool: it translates an annotated program into input for the Viper framework, which is then used as verification back-end. The paper discusses the different programming languages and features for which VerCors provides verification support. It also discusses how the tool internally has been reorganised to become easily extendible, and to improve the connection and interaction with Viper. In addition, we also introduce two tools built on top of VerCors, which support correctness-preserving transformations of verified programs. Finally, we discuss how the VerCors verifier has been used on a range of realistic case studies.
        \item DOI: \href{http://dx.doi.org/10.1007/978-3-031-65630-9_1}{10.1007/978-3-031-65630-9\_1}
    \end{cvitems}
}


\cventry
{International Conference on Fundamental Approaches to Software Engineering 2024} % Degree
{First Steps towards Deductive Verification of LLVM IR} % Institution
{Luxembourg City, Luxembourg} % Location
{April 8–11, 2024} % Date(s)
{
    \begin{cvitems} % Description(s) bullet points
        \item Authors: Dré van Oorschot, Marieke Huisman, \"Omer \c{S}akar
        \item Abstract: Over the last years, deductive program verifiers have substantially improved, and their applicability on non-trivial applications has been demonstrated. However, a major bottleneck is that for every new programming language, a new deductive verifier has to be built. This paper describes the first steps in a project that aims to address this problem, by language-agnostic support for deductive verification: Rather than building a deductive program verifier for every programming language, we develop deductive program verification technology for a widely-used intermediate representation language (LLVM IR), such that we eventually get verification support for any language that can be compiled into the LLVM IR format. Concretely, this paper describes the design of VCLLVM, a prototype tool that adds LLVM IR as a supported language to the VerCors verifier. We discuss the challenges that have to be addressed to develop verification support for such a low-level language. Moreover, we also sketch how we envisage to build verification support for any specified source program that can be compiled into LLVM IR on top of VCLLVM.
        \item DOI: \href{https://doi.org/10.1007/978-3-031-57259-3_15}{10.1007/978-3-031-57259-3\_15}
    \end{cvitems}
}


\cventry
{International Conference on Tools and Algorithms for the Construction and Analysis of Systems 2022} % Degree
{Alpinist: an Annotation-Aware GPU Program} % Institution
{Munich, Germany} % Location
{April 2–7, 2022} % Date(s)
{
	\begin{cvitems} % Description(s) bullet points
		\item Authors: \"Omer \c{S}akar, Mohsen Safari, Marieke Huisman, Anton Wijs
		\item Abstract: Over the last years, deductive program verifiers have substantially improved, and their applicability on non-trivial applications has been demonstrated. However, a major bottleneck is that for every new programming language, a new deductive verifier has to be built. This paper describes the first steps in a project that aims to address this problem, by language-agnostic support for deductive verification: Rather than building a deductive program verifier for every programming language, we develop deductive program verification technology for a widely-used intermediate representation language (LLVM IR), such that we eventually get verification support for any language that can be compiled into the LLVM IR format. Concretely, this paper describes the design of VCLLVM, a prototype tool that adds LLVM IR as a supported language to the VerCors verifier. We discuss the challenges that have to be addressed to develop verification support for such a low-level language. Moreover, we also sketch how we envisage to build verification support for any specified source program that can be compiled into LLVM IR on top of VCLLVM.
		\item DOI: \href{https://doi.org/10.1007/978-3-030-99527-0_18}{10.1007/978-3-030-99527-0\_18}
	\end{cvitems}
}
\end{cventries}

\cvsection{Presentations}

\begin{cventries}
    
    \cventry
    {Slides: \url{\web/pdf/master_colloquium.pdf}}
    {Master's colloquium (Presentation of Master's thesis)}
    {University of Twente, Enschede, Netherlands}
    {22 April 2020}
    {}
    
    \cventry
    {Slides: \url{\web/pdf/lean_colloquium.pdf}}
    {Software Verification with LEAN.}
    {University of Twente, Enschede, Netherlands}
    {2021}
    {Joint talk at the FMT colloquium with Bob Rubbens, Lukas Armborst, Ömer Şakar.}
    
    \cventry
    {Slides: \url{\web/pdf/lean_colloquium.pdf}}
    {Alpinist: an Annotation-Aware GPU Program Optimizer.}
    {International Conference on Tools and Algorithms for the Construction and Analysis of Systems 2022}
    {2022}
    {Joint talk at the FMT colloquium with Bob Rubbens, Lukas Armborst, Ömer Şakar.}
    
    \cventry
    {Slides: \url{\web/pdf/codeplay_visit_slides.pdf}}
    {VerCors+SYCL}
    {Edinburgh, Scotland}
    {2025}
    {Presenting VerCors and its SYCL support to Codeplay Software Ltd. } 
    
\end{cventries}